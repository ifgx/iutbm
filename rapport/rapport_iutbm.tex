\documentclass[a4paper, 11pt, oneside]{report}

% BASE PACKAGES
\usepackage[utf8]{inputenc}
\usepackage[francais]{babel}
\usepackage{graphicx}\graphicspath{{images/}}
\usepackage{calc}
\usepackage[T1]{fontenc}
\usepackage[top=2.5cm, bottom=2cm, left=2.5cm, right=2cm]{geometry}
\usepackage{color}
\usepackage{multirow}
\usepackage{subfig}

\usepackage[absolute]{textpos}
% PYTHON CODE INCLUSION
\usepackage{listings}
% HACK FOR SUMMARY != TOC
\usepackage[tight]{shorttoc}
% LOCAL PACKAGES
\usepackage{phdthesis}
\usepackage{packages/simpleglossary}
\usepackage{packages/bigcenter}
\usepackage{hyperref}
\usepackage[numbers, super, square, comma, sort&compress]{natbib} % biblio
%____________________________________________________________________________
%% FOR LaTeX FONTS, REFER TO: http://www.tug.dk/FontCatalogue/
% SERIF FONT SELECTION
%\usepackage{tgpagella}               % <3<3
%\usepackage[light,condensed]{iwona}  % OK
%\usepackage{palatino}                % <3<3
%\usepackage{lmodern}                 % <3
% SANS-SERIF FONT SELECTION
%\usepackage{times}
%\usepackage[condensed,math]{kurier}  % <3<3
%\usepackage{tgadventor}
%\renewcommand*\familydefault{\sfdefault}
% MONOSPACED FONT SELECTION
%\usepackage{courier}
%\usepackage{tgcursor}
%\usepackage{inconsolata}
%____________________________________________________________________________
% SOME TWEAKS
\setlength{\parskip}{8pt}
\setlength{\headheight}{15pt}
\definecolor{grey}{rgb}{0.95,0.95,0.95}
\hypersetup{
  unicode=true,
  colorlinks=true,
  citecolor=black,
  filecolor=black,
  linkcolor=black,
  urlcolor=black,
  pdfauthor={Benoit Marine\\Durosier Florian\\Nowinski David\\Voisin Julien\\Zambelli Cédric},
  pdftitle={Sensibilisation à la complexité et à la recherche opérationnelle},
  pdfcreator={pdftex},
  pdfsubject={Sensibilisation à la complexité et à la recherche opérationnelle},
  pdfkeywords={iutbm, complexité, algorithmique, graphes},
}

\definecolor{dkgreen}{rgb}{0,0.6,0}
\definecolor{gray}{rgb}{0.5,0.5,0.5}
\definecolor{mauve}{rgb}{0.58,0,0.82}

\lstset{tabsize=2,
        frame=single,
        numbers=left,
        numberstyle=\footnotesize,
        basicstyle=\ttfamily,
        backgroundcolor=\color{grey},
        showtabs=false,
        showspaces=false,
        showstringspaces=false,
        tabsize=2,
		language=Python,
		keywordstyle=\color{blue},          % keyword style
        commentstyle=\color{dkgreen},       % comment style
        stringstyle=\color{mauve},          % string literal style
		morekeywords={pour,de,finpour,finsi,si,sinon,tantque,fintantque,supprimer,dans}
        }

\newlength{\imgwidth}

\newcommand\scalegraphics[1]{%
    \settowidth{\imgwidth}{\includegraphics{#1}}%
    \setlength{\imgwidth}{\minof{\imgwidth}{\textwidth}}%
    \includegraphics[width=\imgwidth]{#1}%
}

%____________________________________________________________________________
% TEMP FRONT PAGE
\title{
    IUT Belfort-Montbeliard - projet sous tutelle\\
    2011\\
    \bigskip {\Huge Sensibilisation à la complexité et à la recherche opérationnelle}
}
\author{Benoit Marine\\Durosier Florian\\Nowinski David\\Sid Ali\\Voisin Julien\\Zambelli Cédric}
\date{\today}

%____________________________________________________________________________
% MAIN MATTER
\begin{document}
% BEGIN HACK
\maketitle

\large

\setcounter{page}{2}
% END HACK
\chapter*{Remerciements}
\begin{center}
\vspace{8\baselineskip}
Nous souhaiterions remercier\\
\vspace*{2\baselineskip}
Deschinkel Karine, pour avoir supervisé le projet\\
\vspace{1\baselineskip}
\url{github.com}, pour la mise à disposition gratuite d'un dépot \emph{git}\\
\vspace{1\baselineskip}
L'IUT de Belfort-Montbeliard\\


\end{center}


% hackish, but works
\setlength{\parskip}{0pt}
\shorttableofcontents{Sommaire}{0}
\setlength{\parskip}{8pt}

\chapter*{Remarques}
\subsection*{Quelques remarques concernant ce rapport}
\begin{itemize}
\setlength{\itemsep}{8pt}
  \item{Un sommaire est présent en début de document. La table des matières
  complète se trouve en fin de document.}

  \item{Les termes techniques et spécifiques sont définis dans une note de pied
  de page à leur première occurrence dans le corps de texte.}
\end{itemize}

\clearpage



\chapter*{Introduction}
\addcontentsline{toc}{chapter}{\numberline{}Introduction}
% a bit hackish
\setcounter{footnote}{0}

Le but de ce projet est de créer un logiciel éducatif de sensibilisation à la recherche
opérationnelle. La recherche opérationnelle (aussi appelée aide à la décision) peut être définie
comme l'ensemble des méthodes et techniques rationnelles orientées vers la recherche de
la meilleure façon d'opérer des choix en vue d'aboutir au résultat visé ou au meilleur résultat possible1.

Ce logiciel permettra à des non-initiés (étudiants, entre autres) de s'initier
de manière ludique aux problématiques de la recherche opérationnelle, qui font partie de notre
vie de tous les jours.


\chapter{Cahier des charges}
Le projet s'articulera autour d'une application interactive.
  L'utilisateur aura à sa disposition trois sous-menus, chacun comportant
  plusieurs jeux basés sur des algorithmes de plus en plus difficiles.
  Ces jeux seront classés selon leur niveau de difficulté -- d'un point de
  vue algorithmique~:

Les jeux se dérouleront par étapes.
\begin{enumerate}
    \item Un texte explicatif du problème posé
    \item Le joueur essaye de trouver la solution optimale par lui-même
    \item Le programme compare la solution calculée à celle du joueur
\end{enumerate}

\section{Portabilité et accessibilité}
    Le programme devra être portable sur les principales plateformes
    grand public (\emph{Windows}, \emph{GNU/Linux}, et \emph{Mac OSX}).
    Il devra également être accessible, le public cible étant
    l'utilisateur lambda.

\section{Langue anglaise}
    Afin de permettre une diffusion la plus large possible,
    l'application devra être écrite en anglais.

\section{Thème unifié}
    Afin de rendre l'application ludique, elle devra
    être habillée d'un thème unifié,
    afin de montrer les \emph{problèmes} sous un
    jour \emph{amusant}.

\section{Notion de complexité}
TODO

\section{Menu principal}
    Le menu principal doit permettre d'accéder rapidement
    aux algorithmes développés, en étant le plus simple possible.



\chapter{Présentation des différents problèmes}
\section{Algorithmes en temps polynomiaux}
    \subsection{Le plus court chemin}
        \subsubsection{Description}
        Le problème du \emph{plus court chemin} consiste à trouver
        le plus court trajet permettant de relier deux points donnés.
        \subsubsection{Applications pratiques}
            L'application la plus évidente de ce problème sont les systèmes de type
            GPS. On l'utilise notamment pour permettre un routage internet très efficace 
            des informations en cherchant le parcours le plus efficace.
        \subsubsection{Algorithme}
			L'algorithme implémenté est l'algorithme de \emph{Dijkstra}: 
			Il s'agit de construire progressivement, à partir des données initiales, 
			un sous-graphe dans lequel sont classés les différents sommets par 
			ordre croissant de leur distance minimale au sommet de départ. 
			
    \subsection{Le couplage}
        \subsubsection{Description}
            Le problème du couplage est un problème de répartition \emph{au mieux}.
            En effet, il s'agit de répartir un certain nombre d'objets
            entre plusieurs personnes, chacun ayant une liste d'objets
            préférés, en contentant un maximum de personnes.
        \subsubsection{Applications pratiques}
            Ce problème interesse particulièrement les entreprises
            de distribution, comme \emph{Amazon},
            ou la \emph{Redoute}, qui doivent
            repartir au mieux leurs marchandises.
        \subsubsection{Algorithme}



\section{Algorithmes pseudo-polynomiaux}
    \subsection{Le problème du sac à dos}
        \subsubsection{Description}
            Le \emph{problème du sac à dos} consiste à sélectionner
            des \emph{objets} auxquelles sont associés un couple (valeur, poids).
            Le but étant d'amasser un maximum de valeur, tout en ne dépassant
            pas une limite de poids fixée.
        \subsubsection{Applications pratiques}
            Cette problématique s'applique par exemple aux espaces portuaires.
            Les ports de marchandises utilisent des conteneurs aux formats standardisés,
            on comprend aisément leur nécessité d'optimiser le contenu de leur
            chargements afin de minimiser les coûts de transports.
        \subsubsection{Algorithme}
            On utilisera un algorithme de progrmmation dynamique. Il consiste à créer 
            un tableau que l'on remplira au fur et à mesure selon les cases du niveaux 
            précédents. 



\section{Algorithmes en temps exponentiels}
     \subsection{Le problème du voyageur de commerce}
        \subsubsection{Description}
                Le problème de voyageur de commerce consiste à trouver le plus
                court chemin reliant tous les points d'un graphe pondéré.
                Ce problème n'est (à ce jour) pas solvable en un temps
                polynomial de manière exacte, c'est pourquoi une \emph{heuristique}
                est utilisée ici.
        \subsubsection{Applications pratiques}
            Toutes les entreprises devant effectuer des tournées sont concernées
            par ce type de problèmes, comme par exemple \emph{EDF},
            la \emph{Poste}, ou encore les entreprises de transport routier.
        \subsubsection{Algorithme}
            L'algorithme implémenté est algorithme de type \emph{plus proche voisin} :
            Il consiste à avancer
            de proche en proche, en prenant à chaque itération le plus
            proche voisin du point actuel.


\chapter{Technique}
\section{Langage}
    Le programme est entièrement codé en Python,
    et utilise la bibliothèque \emph{pygame}.
    \subsection{Python}
        Le \emph{Python} est un langage de programmation de haut niveau\footnote{Qui fait
        abstraction des détails de la machine sur laquelle il s'exécute} interprété
        \footnote{Au contraire des langages compilés, le python est évalué
        lignes par lignes par un \emph{interpréteur}, au lieu d'être exécuté directement
        par la machine}. Il fonctionne sur la plupart des plateformes informatiques.
    \subsection{pygame}
        pygame est une bibliothèque écrite en \emph{Python} initialement prévue pour
        le développement de jeux vidéo. Elle permet de programmer la partie multimédia
        (graphismes, sons, entrées des périphériques (clavier, couris, joystick,...),
        système de fenêtrage, ...) rapidement et facilement.

\section{Outils}
    Tout les outils utilisés, sans exception sont disponibles sous
    licence libre.
    \subsection{git}
        \emph{git} fut utilisé en gestionnaire de versions,
        afin de simplifier le travail collaboratif.
        Il est également un excellent indicateur de la répartition
        de la charge de travail.
    \subsection{pep8, pylint}
        \emph{pep8}, et \emph{pylint} sont tout les deux
        des outils d'analyse statique de code \emph{Python}.
        \subsubsection{pep8}
            \emph{pep8} se charge de vérifier si le code qui lui
            est fourni en entrée respecte bien les conventions de codage
            python de la norme \emph{pep8} (le logiciel est homonyme).
        \subsubsection{pytlint} pylint est quant à lui un outil d'analyse
            statique pur et dur. Il vérifie grossièrement la sémantique du code,
            ainsi que sa présentation, et son respect des bonnes pratiques.
            Par exemple, si les classes\footnote{Objet logiciel representant une entité}
            possèdent bien des commentaires les décrivant, ou encore si les modules
            \footnote{Un module est une \emph{brique} logicielle permettant
            de s'appuyer sur l'existant, et donc de ne pas réinventer la roue}
            importés sont bien utilisés, si les variables ont des noms corrects, etc.
        \subsubsection{vim}
            Ce projet fût également l'occasion pour certains de découvrir \emph{vim},
            un puissant éditeur de texte.

\section{Choix du thème}
    Le programme adopte un thème unifié autour du monde de la pizza. Dans tous les mini-jeux, on pourra
    ainsi retrouver soit une livraison de pizza, soit une distribution de pizza. Cette unification permet à
    l'utilisateur de mieux se retrouver au sein des différents mini-jeux. De plus, elle procure aux mini-jeux
    une certaine cohérence qui aide à la compréhension.

\section{Menu principal}
	Le menu principal est constitué d'une barre de titre en haut de l'écran, et de cinq boutons orange
	 dans la partie inférieure. Quatre de ces boutons permettent d'accéder aux différents jeux proposés
	 d'un simple clic de souris, et le dernier permet de quitter l'application.

\section{Spécificités} % FIXME
    Les instances de problèmes présents dans l'application
    sont dynamiquement générées, afin de renouveler
    le challenge à chaque démarrage.


\chapter{Réalisation}
\section{Algorithmes}
    \subsection{Voyageur de commerce}
        \subsubsection{Génération des données}
            La génération des données se fait en deux temps :
            \begin{enumerate}
                \item Génération de points à des coordonnées aléatoires
                \item Génération de la matrice des distances
            \end{enumerate}
        \subsubsection{Interface}
            L'utilisateur sélectionne les points par simple clic,
            et peut annuler ses actions par clic droit.
            A chaque point sélectionné, le programme
            affiche la distance totale entre les points reliés.
            Dès lors qu'un point a été sélectionné, le programme
            affiche la distance minimum calculée, au-dessus de
            la distance actuelle parcourue par l'utilisateur.
            Lorsque tout les points ont étés reliés entre eux,
            le programme raccorde automatiquement le premier
            et le dernier point, afin de former une boucle.
        \subsubsection{Résolution}
            Lors de la sélection du premier point,
            le programme calcule à l'aide de l'algorithme
            du \emph{plus proche voisin} le chemin le plus court.
            Une fois que l'utilisateur a relié tout les points
            entre eux, le programme affiche la solution calculée
            en surimpression de celle proposée par l'utilisateur.


	\subsection{Couplage}
		\paragraph{Contexte}
			Le probleme de couplage est modélisé par un jeu de choix.
			Il y a plusieurs clients, avec leurs préférences qui ont
			 le choix entre plusieurs pizzas, et le joueur doit 
			 satisfaire un maximum de clients.
		\paragraph{Moteur}
			Au demarrage de l'application, le nombre de pizzas et 
			 de clients est généré aléatoirement, dans un intervalle 
			 assez bas, par soucis de complexité.
			Les préférences de chaque client sont aussi générés aléatoirement,
			 mais dans l'optique qu'un client ait au minimum une préférences.
		\paragraph{Interface}
		    L'écran est divisé en trois colonnes :
		    \begin{itemize}
		        \item[à gauche] les préférences de chancun des clients
		        \item[au centre] les clients
		        \item[à droite] les pizzas disponibles
		    \end{itemize}

		L'utilisateur \emph{sélectionne} un client en cliquant dessus.
		    Une fois un client sélectionné, il est possible
		    d'associer une pizza à ce client, en cliquant dessus.
		    Si la pizza est déjà attribuée, ou si le client a déjà
		    une pizza qui lui est affectée, un message d'erreur s'affiche.
		    Si ni le client n'a de pizza, ni la pizza de client, l'association
		    se fait, et est représentée par un trait.
		L'utilisateur a la possibilité d'annuler ses choix en cliquant avec le bouton
		    droit de la souris (ce qui annule la dernière action).
		\paragraph{Résolution}
			L'algorithme permet d'avoir la solution comportant un maximum de clients
			 satisfait et de verifier si la solution proposée par le joueur correspond
			 a la meilleure solution possible.
 			Si l'utilisateur n'a pas trouvé la solution optimale, la solution s'affiche
			 en surimpression sur le choix du joueur.



    \subsection{Sac à Dos}
        \paragraph{Contexte}
        On pourra modéliser ce problème par un tableau contenant les objets et leur
        valeur en vis-à-vis. L'utilisateur fera glisser les objets du tableau vers
        le sac à dos. Un indicateur permettra de visualiser le poids actuel total
        des objets dans le sac, le poids maximal que peut supporter le sac, et le
        poids restant (différence entre ces deux résultats).

        On pourra scénariste ce jeu avec, par exemple, une histoire où un personnage
        doit quitter son château assiégé en emportant un maximum d'objets de valeur.


            Dans ce mini-jeu, le but du joueur est d'aider un pizzaiolo à
            sélectionner des ingrédients à transporter à un concours de pizza.
            Il ne peut pas prendre tout ce qu'il souhaite et doit faire des
            choix entre qualité et quantité : de bons ingrédients lui donnera
            plus de points par le jury, cependant la pizza présenté a un poids
            limité au gramme près.
        \paragraph{Moteur}
            Au démarrage de l'application, une liste d'ingrédients est créée.
            Pour respecter une certaine logique, on ne peux pas générer des
            chiffre aléatoirement. (Il semblerait assez illogique qu'une
            garniture au fromage pèse plusieurs centaines de grammes par
            exemple.)
        \paragraph{Interface}
            L'interface se présente en 2 colonnes. La première, à gauche,
            contient les objets disponible pour la pizza. La seconde, à droite,
            montre le contenu actuel de la pizza que le joueur à choisi.
            Un clic sur une icone d'un ingrédients permet de l'échanger entre
            les deux colonnes.
            Lorsque l'utilisateur a terminé, il clique sur le bouton de
            validation pour savoir si il a trouvé la solution correcte.
        \paragraph{Résolution}
            La résolution du problème s'effectue en utilisant un algorithme
            détaillé par notre professeur.


\subsection{Problème du plus court chemin}
        L'utilisateur pourra cliquer sur les différentes villes, tour à tour,
        pour déplacer le personnage. Un compteur kilométrique indiquera la
        distance parcourue.

        Le joueur peut effectuer un clic droit pour annuler son dernier
        mouvement.

        Le jeu continue tant que le joueur n'a pas trouvé le chemin
        le plus court. Il peut aussi cliquer sur \og solution\fg, qui
        affichera le chemin segment par segment.

        \paragraph{Contexte}
            Le but du jeu est de livrer une pizza en effectuant le moins de kilomètres possible.
        \paragraph{Moteur}
            Au démarrage du jeu, le programme génère un graphe
            représentant des villes. Ces villes sont reliées par des arcs pondérés.
            Le poids des arcs corresponds à la distance entre chaque villes.
            Il affiche les villes et les liaisons.

        \paragraph{Interface}
            L'utilisateur sélectionne les villes par simple clic,
            et peut annuler ses actions par clic droit.
            Pour revenir au début il suffit de cliquer sur la ville de départ.
            A chaque ville sélectionnée, le programme
            affiche en temps réel la distance totale parcourue.

        \paragraph{Résolution}
            Lorsque la ville de début et la ville finale sont reliées,
            on appelle la méthode de résolution par l'algorithme de \emph{Dijkstra}.

            Le programme affiche ensuite si le joueur à trouvé la solution optimale.

		\paragraph{Algorithme de Dijkstra} ~

			\begin{lstlisting}
liste_sommets = [+oo pour i dans sommets]
S(liste_sommets[0]) = 0
tant que taille(liste_sommets) != 0; faire
    element_courant = min(liste_sommets)
    pour chaque element_voisin dans liste_sommets; faire
        mettre_a_jour(element_voisin)
    fin pour
    supprimer element_courant de liste_sommets
fin tant que
			\end{lstlisting}



\chapter*{Conclusion}
\addcontentsline{toc}{chapter}{\numberline{}Conclusion}
Dans le cadre de notre projet sous tutelle, nous avons réalisé un logiciel ludique de sensibilisation
à la recherche opérationnelle, permettant au débutant de s'initier à des problèmes
complexe de manière amusante.

Ce projet fut pour nous l'occasion de travailler en groupe, et d'en découvrir au passage
les outils inhérents. Il nous a également permis de découvrir le langage
\emph{Python}, ainsi que la bibliothèque \emph{Pygame}.
Mais le point le plus important est sûrement que nous avons pu appliquer
bon nombre de préceptes appris lors de nos deux années passées en DUT.

Dans un souci de réalisme au niveau des délais, nous avons sélectionné de
manière arbitraire les problèmes traités, en choisisant ceux qui nous
semblaient les plus intéressants, avec des applications
concrètes instinctives et évidentes dans la vie réelle
(Comme par exemple le problème de plus court chemin dans les systèmes GPS).
On peut citer, entre autres, au rang des absents les problèmes de colorations de graphes.

Parmi les problèmes rencontrés, les plus importants ont eu trait
non pas aux côtés techniques, mais plutôt du côté humain,
notamment au niveau de la répartition de la charge de travail.


\clearpage

% hackish, but works
\setlength{\parskip}{0pt}
\tableofcontents
\addcontentsline{toc}{chapter}{\numberline{}Table des matières}
\setlength{\parskip}{8pt}

\clearpage
\addcontentsline{toc}{chapter}{\numberline{}Table des figures}
\listoffigures
\clearpage

\begin{thebibliography}{9}

\bibitem{ref.yrr}
    Y. Nobert, R.Ouellet et R. Parent,
    La recherche opérationnelle,
    Gaëtan Morin,
    1995.

\bibitem{wp-complex}
    Wikipedia,
    \emph{Théorie de la complexité des algorithmes},
    \url{https://fr.wikipedia.org/wiki/Th\%C3\%A9orie_de_la_complexit\%C3\%A9_des_algorithmes},
    31 janvier 2011

\end{thebibliography}


\end{document}

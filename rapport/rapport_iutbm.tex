\documentclass[a4paper, 11pt, oneside]{report}

% BASE PACKAGES
\usepackage[utf8]{inputenc}
\usepackage[francais]{babel}
\usepackage{graphicx}\graphicspath{{images/}}
\usepackage{calc}
\usepackage[T1]{fontenc}
\usepackage[top=2.5cm, bottom=2cm, left=2.5cm, right=2cm]{geometry}
\usepackage{color}
\usepackage{multirow}
\usepackage{subfig}

\usepackage[absolute]{textpos}
% PYTHON CODE INCLUSION
\usepackage{listings}
% HACK FOR SUMMARY != TOC
\usepackage[tight]{shorttoc}
% LOCAL PACKAGES
\usepackage{phdthesis}
\usepackage{packages/simpleglossary}
\usepackage{packages/bigcenter}
\usepackage{hyperref}
\usepackage[numbers, super, square, comma, sort&compress]{natbib} % biblio
%____________________________________________________________________________
%% FOR LaTeX FONTS, REFER TO: http://www.tug.dk/FontCatalogue/
% SERIF FONT SELECTION
%\usepackage{tgpagella}               % <3<3
%\usepackage[light,condensed]{iwona}  % OK
%\usepackage{palatino}                % <3<3
%\usepackage{lmodern}                 % <3
% SANS-SERIF FONT SELECTION
%\usepackage{times}
%\usepackage[condensed,math]{kurier}  % <3<3
%\usepackage{tgadventor}
%\renewcommand*\familydefault{\sfdefault}
% MONOSPACED FONT SELECTION
%\usepackage{courier}
%\usepackage{tgcursor}
%\usepackage{inconsolata}
%____________________________________________________________________________
% SOME TWEAKS
\setlength{\parskip}{8pt}
\setlength{\headheight}{15pt}
\definecolor{grey}{rgb}{0.95,0.95,0.95}
\hypersetup{
  unicode=true,
  colorlinks=true,
  citecolor=black,
  filecolor=black,
  linkcolor=black,
  urlcolor=black,
  pdfauthor={Benoit Marine\\Durosier Florian\\Nowinski David\\Voisin Julien\\Zambelli Cédric},
  pdftitle={Sensibilisation à la complexité et à la recherche opérationnelle},
  pdfcreator={pdftex},
  pdfsubject={Sensibilisation à la complexité et à la recherche opérationnelle},
  pdfkeywords={iutbm, complexité, algorithmique, graphes},
}

\definecolor{dkgreen}{rgb}{0,0.6,0}
\definecolor{gray}{rgb}{0.5,0.5,0.5}
\definecolor{mauve}{rgb}{0.58,0,0.82}

\lstset{tabsize=2,
        frame=single,
        numbers=left,
        numberstyle=\footnotesize,
        basicstyle=\ttfamily,
        backgroundcolor=\color{grey},
        showtabs=false,
        showspaces=false,
        showstringspaces=false,
        tabsize=2,
		language=Python,
		keywordstyle=\color{blue},          % keyword style
        commentstyle=\color{dkgreen},       % comment style
        stringstyle=\color{mauve},          % string literal style
		morekeywords={pour,de,finpour,finsi,si,sinon,tantque,fintantque,supprimer,dans}
        }

\newlength{\imgwidth}

\newcommand\scalegraphics[1]{%
    \settowidth{\imgwidth}{\includegraphics{#1}}%
    \setlength{\imgwidth}{\minof{\imgwidth}{\textwidth}}%
    \includegraphics[width=\imgwidth]{#1}%
}

%____________________________________________________________________________
% TEMP FRONT PAGE
\title{
    IUT Belfort-Montbeliard - projet sous tutelle\\
    2011\\
    \bigskip {\Huge Sensibilisation à la complexité et à la recherche opérationnelle}
}
\author{Benoit Marine\\Durosier Florian\\Nowinski David\\Sid Ali\\Voisin Julien\\Zambelli Cédric}
\date{\today}

%____________________________________________________________________________
% MAIN MATTER
\begin{document}
% BEGIN HACK
\maketitle

\large

\setcounter{page}{2}
% END HACK
\chapter*{Remerciements}
\begin{center}
\vspace{8\baselineskip}
Nous souhaiterions remercier\\
\vspace*{2\baselineskip}
Deschinkel Karine, pour avoir supervisé le projet\\
\vspace{1\baselineskip}
\url{github.com}, pour la mise à disposition gratuite d'un dépot \emph{git}\\
\vspace{1\baselineskip}
L'IUT de Belfort-Montbeliard\\


\end{center}


% hackish, but works
\setlength{\parskip}{0pt}
\shorttableofcontents{Sommaire}{0}
\setlength{\parskip}{8pt}

\chapter*{Remarques}
\subsection*{Quelques remarques concernant ce rapport}
\begin{itemize}
\setlength{\itemsep}{8pt}
  \item{Un sommaire est présent en début de document. La table des matières
  complète se trouve en fin de document.}

  \item{Les termes techniques et spécifiques sont définis dans une note de pied
  de page à leur première occurrence dans le corps de texte.}
\end{itemize}

\clearpage



\chapter*{Introduction}
\addcontentsline{toc}{chapter}{\numberline{}Introduction}
% a bit hackish
\setcounter{footnote}{0}

Le but de ce projet tuteuré est de créer un logiciel éducatif de sensibilisation à la recherche
 opérationnelle. Ce logiciel permettra à des non-initiés (étudiants, entre autres) de s'initier
 de manière ludique aux problématiques de la recherche opérationnelle, qui font partie de notre
 vie de tous les jours.




\chapter{Cahier des charges}
Le projet s'articulera autour d'une application interactive.
L'application proposera un menu clair, permettant
d'acceder rapidement aux problèmes proposés.

Les jeux se dérouleront par étapes.
\begin{enumerate}
    \item Un texte explicatif du problème posé
    \item Le joueur essaye de trouver la solution optimale par lui-même
    \item Le programme compare la solution calculée à celle du joueur
\end{enumerate}

\section{Portabilité et accessibilité}
    Le programme devra être portable sur les principales plateformes
    grand public (\emph{Windows}, \emph{GNU/Linux}, et \emph{Mac OSX}).
    Il devra également être accessible, le public cible n'étant
    pas forcément familier de l'outil informatique.

\section{Langue anglaise}
    Afin de permettre une diffusion la plus large possible,
    l'application devra être écrite en anglais.

\section{Thème unifié}
    Afin de rendre l'application ludique, elle devra
    être habillée d'un thème unifié,
    afin de montrer les \emph{problèmes} sous un
    jour \emph{amusant}.

\section{Menu principal}
    Le menu principal doit permettre d'accéder rapidement
    aux algorithmes développés, en étant le plus simple possible.

\section{Notion de complexité}
La théorie de la complexité s'attache à connaître la difficulté (ou la
complexité) d'une réponse par algorithme à un problème, dit algorithmique, posé
de façon mathématique. Pour la définir, il faut présenter les concepts de
problèmes algorithmiques, de réponses algorithmiques aux problèmes, et la
complexité des problèmes algorithmiques.~\cite{wp-complex} 
Cette complexité va permettre d'évaluer différents algorithmes qui vont 
résoudre le même problème. 
 Il existe de nombreux algorithmes pour trier un tableau. On va présenter deux 
de ces algorithmes l'un étant le tri par insertion, l'autre étant le tri rapide dit \emph{quik sort}.
Le tri par insertion a une complexité polynomiale qui est égale à $O(n)$ . 
Le tri rapide lui a une complexité linéarithmique qui est égale à $O(n \times \log(n) )$ . 
Le tri rapide est beaucoup plus efficace que le tri à bulle sur de  grand nombre. 
Toutefois sur des tableux des moins de 15 élément le tri par insertion est plus efficace. 
Ici la complexité permet de choisir un algorithme selon les type de données que l'on va traiter.

\section{Notion de recherche opérationnelle}
La recherche opérationnelle (également appellée \emph{aide à la decision})
est la recherche de la meilleure façon d'aboutir au milleur résultat
possible, ou au résultat visé.


\chapter{Présentation des différents problèmes}
\section{Algorithmes en temps polynomiaux}
    \subsection{Le plus court chemin}
        \subsubsection{Description}
        Le problème du \emph{plus court chemin} consiste à trouver
        le plus court trajet permettant de relier deux points donnés dans un graphe pondéré.
        \subsubsection{Applications pratiques}
            L'application la plus évidente de ce problème sont les systèmes de type
            GPS, ou de manière générale dans tous les problèmes necessitant
            un parcours optimal, comme le réseau internet, pour la commutation de
            paquets, ou encore les livreurs.
        \subsubsection{Algorithme} % TODO : rapide presentation de l'algo
	L'algorithme implémenté est l'algorithme de \emph{Dijkstra}:

    \subsection{Le couplage}
        \subsubsection{Description}
            Le problème du couplage est un problème de répartition \emph{au mieux}.
            En effet, il s'agit de répartir un certain nombre d'objets
            entre plusieurs personnes, chacun ayant une liste d'objets
            préférés, en contentant un maximum de personnes.
        \subsubsection{Applications pratiques}
            Ce problème interesse particulièrement les entreprises
            de distribution, comme \emph{Amazon},
            ou la \emph{Redoute}, qui doivent
            repartir au mieux leurs marchandises.
        \subsubsection{Algorithme}
		Nous pouvons penser le problème du couplage comme un graphe,
		 où les sommets représentent les personnes et les objets,
		 et les arrêtes, les préférences des personnes.
		L'algorithme implémenté est donc un algorithme permettant
		 de résoudre ce genre de graphe.
		Les arrêtes ont des capacités de 1 car chaque préférences est unique
		 pour chaque client.
		Le but est de contenter chaque préférences, donc nous utilisons un
		 algorithme permettant de résoudre un problème de flot maximal
		 dans un graphe.



\section{Algorithmes pseudo-polynomiaux}
    \subsection{Le problème du sac à dos}
        \subsubsection{Description}
            Le \emph{problème du sac à dos} consiste à sélectionner
            des \emph{objets} auxquelles sont associés un couple (valeur, poids).
            Le but étant d'amasser un maximum de valeur, tout en ne dépassant
            pas une limite de poids fixée.
        \subsubsection{Applications pratiques}
            Cette problématique s'applique par exemple aux espaces portuaires.
            Les ports de marchandises utilisent des conteneurs aux formats standardisés,
            on comprend aisément leur nécessité d'optimiser le contenu de leur
            chargements afin de minimiser les coûts de transports.
        \subsubsection{Algorithme}
            On utilisera un algorithme de progrmmation dynamique. La programmation dynamique s'appuie sur un principe simple :
	    toute solution optimale s'appuie elle-même sur des sous-problèmes résolus localement de façon optimale.



\section{Algorithmes en temps exponentiels}
     \subsection{Le problème du voyageur de commerce}
        \subsubsection{Description}
                Le problème de voyageur de commerce consiste à trouver le plus
                court parcours reliant tous les points d'un graphe pondéré.
                Ce problème n'est (à ce jour) pas solvable en un temps
                polynomial de manière exacte pour de très grands graphes.
                Actuellement, avec de très gros calculateurs, il est possible
                de résoudre de manière exacte ce problème pour des graphes
                de la taille d'un pays (environ 15.000 villes).

                Pour des graphes de tailles superieure, ou tout simplement
                si la puissance de calcul est moindre, on utilise des heuristiques,
                comme par exemple le plus proche voisin ou encore l'algorithme de Christophide.

                On utilise également des meta-heuristiques (k-opt, v-opt, \ldots),
                qui sont en fait des combinaisons
                d'heuristiques, permettant d'obtenir des résultats encore plus intéressants,
                sans alourdir de manière excessive le temps de calcul.

        \subsubsection{Applications pratiques}
            Toutes les entreprises devant effectuer des tournées sont concernées
            par ce type de problèmes, comme par exemple \emph{EDF},
            la \emph{Poste}, ou encore les entreprises de transport routier.
        \subsubsection{Algorithme}
            L'heuristique choisi ici est l'algorithme du plus proche voisin,
            en raison de sa facilité d'implémentation, et sa relative efficacité.
            Il consiste à avancer de proche en proche, en prenant à chaque itération le plus
            proche voisin du point actuel, jusqu'à ce que tous les points aient
            été parcourus.


\chapter{Technique}
\section{Langage}
    Le programme est entièrement codé en Python,
    et utilise la bibliothèque \emph{pygame}.
    \subsection{Python}
        Le \emph{Python} est un langage de programmation de haut niveau\footnote{Qui fait
        abstraction des détails de la machine sur laquelle il s'exécute} interprété
        \footnote{Au contraire des langages compilés, le python est évalué
        lignes par lignes par un \emph{interpréteur}, au lieu d'être exécuté directement
        par la machine}. Il fonctionne sur la plupart des plateformes informatiques.
    \subsection{pygame}
        pygame est une bibliothèque écrite en \emph{Python} initialement prévue pour
        le développement de jeux vidéo. Elle permet de programmer la partie multimédia
        (graphismes, sons, entrées des périphériques (clavier, couris, joystick,...),
        système de fenêtrage, ...) rapidement et facilement).

\section{Outils}
    Tout les outils utilisés, sans exception sont disponibles sous
    licence libre.
    \subsection{git}
        \emph{git} fût utilisé en gestionnaire de version,
        afin de simplifier le travail collaboratif.
        Il est également un excellent indicateur de la répartition
        de la charge de travail.
    \subsection{pep8, pylint}
        \emph{pep8}, et \emph{pylint} sont tout les deux
        des outils d'analyse statique de code \emph{Python}
        \subsubsection{pep8}
            \emph{pep8} se charge de vérifier si le code qui lui
            est fourni en entrée respecte bien les convention de codage
            python de la norme \emph{pep8} (le logiciel est homonyme).
        \subsubsection{pytlint} pylint est quant à lui un outil d'analyse
            statique pur et dur. Il vérifie grossièrement la sémantique du code,
            ainsi que sa présentation, et son respect des bonnes pratiques.
            Par exemple, si les classes\footnote{Objet logiciel representant une entité}
            possèdent bien des commentaires les décrivant, ou encore si les modules
            \footnote{Un module est une \emph{brique} logicielle permettant
            de s'appuyer sur l'existant, et donc de ne pas réinventer la roue}
            importés sont bien utilisés, si les variables ont des noms corrects, ...
        \subsubsection{vim}
            Ce projet fût également l'occasion pour certain de découvrir \emph{vim},
            un puissant éditeur de texte modal.

\section{Choix du thème}
Le programme adopte un thème unifié autour du monde de la pizza. Dans tous les mini-jeux, on pourra
ainsi retrouver soit une livraison de pizza, soit une distribution de pizza. Cette unification permet à
l'utilisateur de mieux se retrouver au sein des différents mini-jeux. De plus, elle procure aux mini-jeux
une certaine cohérence qui aide à la compréhension.

\section{Menu principal}
    Le menu principal est un menu circulaire,
    permettant de choisir les algorithmes, et de
    quitter le programme.


\chapter{Réalisation}
\section{Langage}
    Le programme est entièrement codé en Python,
    et utilise la bibliothèque \emph{pygame}.
    \subsection{Python}
        Le \emph{Python} est un langage de programmation de haut niveau\footnote{Qui fait
        abstraction des détails de la machine sur laquelle il s'exécute} interprété
        \footnote{Au contraire des langages compilés, le python est évalué
        lignes par lignes par un \emph{interpréteur}, au lieu d'être exécuté directement
        par la machine}. Il fonctionne sur la plupart des plateformes informatiques.
    \subsection{pygame}
        pygame est une bibliothèque écrite en \emph{Python} initialement prévue pour
        le développement de jeux vidéo. Elle permet de programmer la partie multimédia
        (graphismes, sons, entrées des périphériques (clavier, couris, joystick,...),
        système de fenêtrage, ...) rapidement et facilement).

\section{Outils}
    Tout les outils utilisés, sans exception sont disponibles sous
    licence libre.
    \subsection{git}
        \emph{git} fût utilisé en gestionnaire de version,
        afin de simplifier le travail collaboratif.
        Il est également un excellent indicateur de la répartition
        de la charge de travail.
    \subsection{pep8, pylint}
        \emph{pep8}, et \emph{pylint} sont tout les deux
        des outils d'analyse statique de code \emph{Python}
        \subsubsection{pep8}
            \emph{pep8} se charge de vérifier si le code qui lui
            est fourni en entrée respecte bien les convention de codage
            python de la norme \emph{pep8} (le logiciel est homonyme).
        \subsubsection{pytlint} pylint est quant à lui un outil d'analyse
            statique pur et dur. Il vérifie grossièrement la sémantique du code,
            ainsi que sa présentation, et son respect des bonnes pratiques.
            Par exemple, si les classes\footnote{TODO} possèdent bien
            des commentaires les décrivant, ou encore si les modules
            \footnote{Un module est une \emph{brique} logicielle permettant
            de s'appuyer sur l'existant, et donc de ne pas réinventer la roue}
            importés sont bien utilisés, si les variables ont des noms corrects, ...
        \subsubsection{vim}
            Ce projet fût également l'occasion pour certain de découvrir \emph{vim},
            un puissant éditeur de texte modal.

\section{Menu principal}
    Le menu principal est un menu circulaire,
    permettant de choisir les algorithmes, et de
    quitter le programme.

\section{Algorithmes}
    \subsection{Voyageur de commerce}
        \paragraph{Moteur}
            Au démarrage du jeu, le programme génère un nuage
            de points (entièrement paramétrable en interne)
            répartis aléatoirement dans la fenêtre, et les affiche.
        \paragraph{Interface}
            L'utilisateur sélectionne les points par simple clic,
            et peut annuler ses actions par clic droit.
            A chaque point sélectionné, le programme
            affiche en temps réel la distance totale entre
            les points reliés.
            Lorsqu'un point a été sélectionné, le programme
            affiche la distance minimum calculée au-dessus de
            la distance actuelle parcourue par l'utilisateur.
            Lorsque tout les points ont étés reliés entre eux,
            le programme raccorde automatiquement le premier
            et le dernier point entre eux.
        \paragraph{Résolution}
            Lors de la sélectionne du premier point,
            le programme calcule à l'aide de l'algorithme
            du plus proche voisin optimisé par permutation
            le chemin le plus court.
            Une fois que l'utilisateur a relié tout les points
            entre eux, le programme affiche la solution calculée
            en surimpression de celle trouvée par l'utilisateur.

            L'algorithme du plus proche voisin consiste à avancer
            de proche en proche, en prenant à chaque itération le plus
            proche voisin du point actuel.
            La permutation permet d'optimiser un peu le résultat trouvé
            par l'algorithme du plus proche voisin.
            Il consiste en une permutation de deux points sur le chemin trouvé.
            Si la permutation réduit la longueur du chemin total,
            elle est conservé, sinon, on essaye une autre permutation,
            et ce, jusqu'à ce que tout les points aient été permutés.
        
	\subsection{Couplage}
		\paragraph{Contexte}
			Le but du jeu est de répartir un certain nombre de pizzas à un certain nombre de
			 clients. Chaque client dresse une liste de ses pizzas préférées; il faut alors,
			 en suivant ces directives, déterminer quelle est la meilleure façon de répartir
			 les pizzas entre les différents clients.

			Au terme de la résolution, il est possible qu'un ou plusieurs clients n'aient pas
			 de pizza, ou au contraire qu'une pizza ne soit distribuée à personne. Le but est
			 alors de satisfaire un maximum de clients.
		\paragraph{Moteur}
			Au démarrage du jeu, l'application génère une liste aléatoire de clients et une
			 liste aléatoire de pizzas. Ensuite, elle génère une liste aléatoire de couples
			 (client, pizza) représentant la liste des préférences de chaque client.
		\paragraph{Interface}
			L'utilisateur est présenté avec une liste de clients à gauche, et une liste
			 de pizzas à droite. En cliquant sur un client, la liste des pizzas préférées
			 du client s'affiche en bas de l'écran.

			L'utilisateur \emph{sélectionne} un client en cliquant dessus. Une fois un client
			 sélectionné, il clique sur la pizza qu'il souhaite attribuer à ce client. Si la
			 pizza est déjà attribuée à quelqu'un ou si le client a déjà une pizza, un message
			 d'erreur s'affiche; sinon, la pizza est associée au client: un trait rouge
			 s'affiche entre les deux.

			L'utilisateur a la possibilité d'annuler ses choix en cliquant avec le bouton
			 droit de la souris (ce qui annule la dernière action).
		\paragraph{Résolution}
			Si l'utilisateur n'a pas trouvé la solution optimale, la solution s'affiche
			 en surimpression sur le choix du joueur.

    \subsection{Sac à Dos}
        \paragraph{Contexte}
            Dans ce mini-jeu, le but du joueur est d'aider un pizzaiolo à
            sélectionner des ingrédients à transporter à un concours de pizza.
            Il ne peut pas prendre tout ce qu'il souhaite et doit faire des
            choix entre qualité et quantité : de bons ingrédients lui donnera
            plus de points par le jury, cependant la pizza présenté a un poids
            limité au gramme près.
        \paragraph{Moteur}
            Au démarrage de l'application, une liste d'ingrédients est créée.
            Pour respecter une certaine logique, on ne peux pas générer des
            chiffre aléatoirement. (Il semblerait assez illogique qu'une
            garniture au fromage pèse plusieurs centaines de grammes par
            exemple.)
        \paragraph{Interface}
            L'interface se présente en 2 colonnes. La première, à gauche,
            contient les objets disponible pour la pizza. La seconde, à droite,
            montre le contenu actuel de la pizza que le joueur à choisi.
            Un clic sur une icone d'un ingrédients permet de l'échanger entre
            les deux colonnes.
            Lorsque l'utilisateur a terminé, il clique sur le bouton de
            validation pour savoir si il a trouvé la solution correcte.
        \paragraph{Résolution}
            La résolution du problème s'effectue en utilisant un algorithme
            détaillé par notre professeur.


\chapter*{Conclusion}
\addcontentsline{toc}{chapter}{\numberline{}Conclusion}
Ce projet fut l'occasion de voir  % FIXME
le développement à plusieurs, et d'utiliser
des outils adaptés.
Il a également permis de découvrir le langage
\emph{Python}, ainsi que la bibliothèque \emph{Pygame}.
Mais plus important, ce projet nous a permis
de découvrir le monde de la \emph{recherche opérationelle},
et de l'\emph{optimisation}, ainsi que de pouvoir y sensibiliser
d'autres personnes.


\clearpage

% hackish, but works
\setlength{\parskip}{0pt}
\tableofcontents
\addcontentsline{toc}{chapter}{\numberline{}Table des matières}
\setlength{\parskip}{8pt}

\clearpage
\addcontentsline{toc}{chapter}{\numberline{}Table des figures}
\listoffigures
\clearpage

\begin{thebibliography}{9}

\bibitem{ref.yrr}
    Y. Nobert, R.Ouellet et R. Parent,
    La recherche opérationnelle,
    Gaëtan Morin,
    1995.

\bibitem{wp-complex}
    Wikipedia,
    \emph{Théorie de la complexité des algorithmes},
    \url{https://fr.wikipedia.org/wiki/Th\%C3\%A9orie_de_la_complexit\%C3\%A9_des_algorithmes},
    31 janvier 2011

\end{thebibliography}


\end{document}

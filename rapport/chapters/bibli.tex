\begin{thebibliography}{9}

\bibitem{ref.yrr}
    Y. Nobert, R.Ouellet et R. Parent,
    La recherche opérationnelle,
    Gaëtan Morin,
    1995.

\bibitem{wp-complex}
    Wikipedia,
    \emph{Théorie de la complexité des algorithmes},
    \url{https://fr.wikipedia.org/wiki/Th\%C3\%A9orie_de_la_complexit\%C3\%A9_des_algorithmes},
    31 janvier 2011
\bibitem{ref.table}
  Wikipedia
  \emph{Complexité, Comparatif}
  \url{http://fr.wikipedia.org/wiki/Th\%C3\%A9orie_de_la_complexit\%C3\%A9_des_algorithmes\#Complexit.C3.A9.2C_comparatif}

\bibitem{ref.dijkstra}
  Wikipedia
  \emph{Algorithme de Dijkstra}
  \url{http://fr.wikipedia.org/wiki/Algorithme_de_Dijkstra}

\bibitem{ref.fordfulkerson}
Calcul de flots
Algorithme de Ford-Fulkerson
\url{http://www-lih.univ-lehavre.fr/~guinand/Enseignement/Graphes/index.html}

\end{thebibliography}

\section{Algorithmes en temps polynomiaux}
    \subsection{Le plus court chemin}
        \subsubsection{Description}
        Le problème du \emph{plus court chemin} consiste à trouver
        le plus court trajet permettant de relier deux points donnés dans un graphe pondéré. 
	Il s'agit d'un algorithme en temps polynomial. Un algorithme en temps polynomial est 
	une algorithme dont le temps d'exécution est proportionnel à un polynôme en fonction des entrées.
	
        \subsubsection{Applications pratiques}
            L'application la plus évidente de ce problème sont les systèmes de type
            GPS, ou de manière générale dans tous les problèmes necessitant
            un parcours optimal, comme le réseau internet, pour la commutation de
            paquets, ou encore les livreurs.
        \subsubsection{Algorithme} % TODO : rapide presentation de l'algo
	Il existe plusieurs algorithmes pour résoudre le problème du plus court chemin. 
	Notamment des algorithmes à origine unique :
\begin{enumerate}
 \item Algorithme de Dijkstra
 \item Algorithme de Ford-Bellman
 \item Algorithme de Gabow
\end{enumerate}
	Ainsi que des algorithmes pour tout couple de sommets :
\begin{enumerate}
 \item Algorithme de Dantzig-Ford
 \item Algorithme de Floyd-Warshall
 \item Algorithme de Johnson
 \item Algorithme du flot maximum
\end{enumerate}

	L'algorithme implémenté est l'algorithme de \emph{Dijkstra}. L'algorithme porte le nom de son inventeur, 
	l'informaticien néerlandais Edsger Dijkstra, et a été publié en 1959. Cet algorithme s'applique sur un graphe pondéré 
	qui de ne possède pas de boucle infinie (pas de pondération négative).

    \subsection{Le couplage}
        \subsubsection{Description}
            Le problème du couplage est un problème de répartition \emph{au mieux}.
            En effet, il s'agit de répartir un certain nombre d'objets
            entre plusieurs personnes ayant chacune une liste d'objets
            préférés, en contentant un maximum de personnes.
        \subsubsection{Applications pratiques}
            Ce problème intéresse particulièrement les entreprises
            de distribution, comme \emph{Amazon},
            ou \emph{La Redoute}, qui doivent
            répartir au mieux leurs marchandises.
        \subsubsection{Algorithme}
		Nous pouvons penser le problème du couplage comme un graphe,
		 où les sommets représentent les personnes et les objets,
		 et les arrêtes, les préférences des personnes.
		L'algorithme implémenté est donc un algorithme permettant
		 de résoudre ce genre de graphe.
		Les arêtes ont des capacités de 1 car chaque préférence est unique
		 pour chaque client.
		Le but est de contenter chaque client, donc nous utilisons un
		 algorithme permettant de résoudre un problème de flot maximal
		 dans un graphe. Nous utiliserons l'algorithme de Ford Fulkerson 
          		qui est un algorithme en temps polynomial.



\section{Algorithmes pseudo-polynomiaux}
    \subsection{Le problème du sac à dos}
        \subsubsection{Description}
            Le \emph{problème du sac à dos} consiste à sélectionner
            des \emph{objets} auxquels sont associés un couple (valeur, poids).
            Le but étant d'amasser un maximum de valeur, tout en ne dépassant
            pas une limite de poids fixée.
        \subsubsection{Applications pratiques}
            Cette problématique s'applique par exemple aux espaces portuaires.
            Les ports de marchandises utilisent des conteneurs aux formats standardisés,
            on comprend aisément leur nécessité d'optimiser le contenu de leurs
            chargements afin de minimiser les coûts de transports.
        \subsubsection{Algorithme}
            On utilisera un algorithme de programmation dynamique. La programmation dynamique s'appuie sur un principe simple :
	    toute solution optimale s'appuie elle-même sur des sous-problèmes résolus localement de façon optimale.



\section{Algorithmes en temps exponentiels}
     \subsection{Le problème du voyageur de commerce}
        \subsubsection{Description}
                Le problème du voyageur de commerce consiste à trouver le plus
                court parcours reliant tous les points d'un graphe pondéré.
                Ce problème n'est (à ce jour) pas solvable en un temps
                polynomial de manière exacte pour de très grands graphes.
                Actuellement, avec de très gros calculateurs, il est possible
                de résoudre de manière exacte ce problème pour des graphes
                de la taille d'un pays (environ 15.000 villes).

                Pour des graphes de taille supérieure, ou tout simplement
                si la puissance de calcul est moindre, on utilise des heuristiques,
                comme par exemple le plus proche voisin ou encore l'algorithme de Christophide.

                On utilise également des méta-heuristiques (k-opt, v-opt, \ldots),
                qui sont en fait des combinaisons
                d'heuristiques, permettant d'obtenir des résultats encore plus intéressants,
                sans alourdir de manière excessive le temps de calcul.

        \subsubsection{Applications pratiques}
            Toutes les entreprises devant effectuer des tournées sont concernées
            par ce type de problème, comme par exemple \emph{EDF},
            \emph{La Poste}, ou encore les entreprises de transport routier.
        \subsubsection{Algorithme}
            L'heuristique choisie ici est l'algorithme du plus proche voisin,
            en raison de sa facilité d'implémentation, et sa relative efficacité.
            Elle consiste à avancer de proche en proche, en prenant à chaque itération le plus
            proche voisin du point actuel, jusqu'à ce que tous les points aient
            été parcourus.

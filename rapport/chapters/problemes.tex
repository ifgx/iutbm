\section{Algorithmes en temps polynomiaux}
    \subsection{Le plus court chemin}
        \subsubsection{Description}
        Le problème du \emph{plus court chemin} consiste à trouver
        le plus court trajet permettant de relier deux points donnés.
        \subsubsection{Applications pratiques}
            L'application la plus évidente de ce problème sont les systèmes de type
            GPS.
        \subsubsection{Algorithme}

    \subsection{Le couplage}
        \subsubsection{Description}
            Le problème du couplage est un problème de répartition \emph{au mieux}.
            En effet, il s'agit de répartir un certain nombre d'objets
            entre plusieurs personnes, chacun ayant une liste d'objets
            préférés, en contentant un maximum de personnes.
        \subsubsection{Applications pratiques}
            TODO
        \subsubsection{Algorithme}
            TODO



\section{Algorithmes pseudo-polynomiaux}
    \subsection{Le problème du sac à dos}
        \subsubsection{Description}
            Le \emph{problème du sac à dos} consiste à sélectionner
            des \emph{objets} auxquelles sont associés un couple (valeur, poids).
            Le but étant d'amasser un maximum de valeur, tout en ne dépassant
            pas une limite de poids fixée.
        \subsubsection{Applications pratiques}
            TODO
        \subsubsection{Algorithme}



\section{Algorithmes en temps exponentiels}
     \subsection{Le problème du voyageur de commerce}
        \subsubsection{Description}
                Le problème de voyageur de commerce consiste à trouver le plus
                court chemin reliant tous les points d'un graphe pondéré.
                Ce problème n'est (à ce jour) pas solvable en un temps
                polynomial de manière exacte, c'est pourquoi une \emph{heuristique}
                est utilisée ici.
        \subsubsection{Applications pratiques}
            Toutes les entreprises devant effectuer des tournées sont concernées
            par ce type de problèmes, comme par exemple \emph{EDF},
            la \emph{Poste}, ou encore les entreprises de transport routier.
        \subsubsection{Algorithme}
            L'algorithme implémenté est algorithme de type \emph{plus proche voisin} :
            Il consiste à avancer
            de proche en proche, en prenant à chaque itération le plus
            proche voisin du point actuel.

            %\begin{algorithmic}
            %    $max_distance = sqrt((max_x - min_x)^2 + (max_y - min_y)^2)$
            %    \FOR{$i = 0 \to nbpts$}
            %    \STATE $minimum\gets max_distance$
            %\end{algorithmic}

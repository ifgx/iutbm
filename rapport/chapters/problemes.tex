\section{Algorithmes en temps polynomiaux}
    \subsection{Le plus court chemin}
        \subsubsection{Description}
        Le problème du \emph{plus court chemin} consiste à trouver
        le plus court trajet permettant de relier deux points donnés.
        \subsubsection{Applications pratiques}
            L'application la plus évidente de ce problème sont les systèmes de type
            GPS. On l'utilise notamment pour permettre un routage internet très efficace 
            des informations en cherchant le parcours le plus efficace.
        \subsubsection{Algorithme}
			L'algorithme implémenté est l'algorithme de \emph{Dijkstra}: 
			Il s'agit de construire progressivement, à partir des données initiales, 
			un sous-graphe dans lequel sont classés les différents sommets par 
			ordre croissant de leur distance minimale au sommet de départ. 
			
    \subsection{Le couplage}
        \subsubsection{Description}
            Le problème du couplage est un problème de répartition \emph{au mieux}.
            En effet, il s'agit de répartir un certain nombre d'objets
            entre plusieurs personnes, chacun ayant une liste d'objets
            préférés, en contentant un maximum de personnes.
        \subsubsection{Applications pratiques}
            Ce problème interesse particulièrement les entreprises
            de distribution, comme \emph{Amazon},
            ou la \emph{Redoute}, qui doivent
            repartir au mieux leurs marchandises.
        \subsubsection{Algorithme}
		Nous pouvons penser le problème du couplage comme un graphe,
		 où les sommets représentent les personnes et les objets,
		 et les arrêtes, les préférences des personnes.
		L'algorithme implémenté est donc un algorithme permettant
		 de résoudre ce genre de graphe.
		Les arrêtes ont des capacités de 1 car chaque préférences est unique 
		 pour chaque client.
		Le but est de contenter chaque préférences, donc nous utilisons un
		 algorithme permettant de résoudre un problème de flot maximal
		 dans un graphe.



\section{Algorithmes pseudo-polynomiaux}
    \subsection{Le problème du sac à dos}
        \subsubsection{Description}
            Le \emph{problème du sac à dos} consiste à sélectionner
            des \emph{objets} auxquelles sont associés un couple (valeur, poids).
            Le but étant d'amasser un maximum de valeur, tout en ne dépassant
            pas une limite de poids fixée.
        \subsubsection{Applications pratiques}
            Cette problématique s'applique par exemple aux espaces portuaires.
            Les ports de marchandises utilisent des conteneurs aux formats standardisés,
            on comprend aisément leur nécessité d'optimiser le contenu de leur
            chargements afin de minimiser les coûts de transports.
        \subsubsection{Algorithme}
            On utilisera un algorithme de progrmmation dynamique. Il consiste à créer 
            un tableau que l'on remplira au fur et à mesure selon les cases du niveaux 
            précédents. 



\section{Algorithmes en temps exponentiels}
     \subsection{Le problème du voyageur de commerce}
        \subsubsection{Description}
                Le problème de voyageur de commerce consiste à trouver le plus
                court chemin reliant tous les points d'un graphe pondéré.
                Ce problème n'est (à ce jour) pas solvable en un temps
                polynomial de manière exacte, c'est pourquoi une \emph{heuristique}
                est utilisée ici.
        \subsubsection{Applications pratiques}
            Toutes les entreprises devant effectuer des tournées sont concernées
            par ce type de problèmes, comme par exemple \emph{EDF},
            la \emph{Poste}, ou encore les entreprises de transport routier.
        \subsubsection{Algorithme}
            L'algorithme implémenté est algorithme de type \emph{plus proche voisin} :
            Il consiste à avancer
            de proche en proche, en prenant à chaque itération le plus
            proche voisin du point actuel.

Ce projet a été enrichissant dans le sens où nous avons acquis certaines
notions du travail collaboratif avec les impératifs et les imprévus que cela entraine.
Nous avons découvert et appris à travers ce projet, différents problèmes
pouvant aboutir au meilleur résultat possible grâce à la recherche opérationnelle.

Cette première experience nous a fait réaliser à quel point le travail
en équipe est important.
En effet, il permet un gain de temps non négligeable dans la mesure où
chacun d'entre nous a travaillé sur des parties bien distinctes du projet.
Le temps gagné a été utilisé afin de mettre en commun le travail de
chacun pour en corriger les éventuels bugs.
Il a donc fallut mettre en place les différents algorithmes de recherche
opérationnelle et les adapter pour les intégrer dans du code \emph{Python}.
Cette phase s'est déroulée plutôt aisément dans la mesure où certaines
personnes étaient déjà adeptes à ce langage.
Pour les autres, ce projet fut l'occasion d'étudier et d'apprendre
plus en détail les facettes du langage utilisé lors du projet,
à savoir \emph{Python}.

Les connaissances des étudiants n'étant pas les même,
la création de groupes hétérogènes couplés à une bonne entente
(le sujet ayant été accepté par l'ensemble du groupe),
ont permis à tous les membres du projet de progresser
et de se préparer à ce que pourrait être la vie en entreprise.

Ce projet fut également pour certains d'entre nous,
l'opportunité de découvrir un outil de travail collaboratif: \emph{git},
un gestionnaire de versions qui fut très utile pour mener
à bien notre projet.

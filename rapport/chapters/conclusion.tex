Dans le cadre de notre projet sous tutelle, nous avons réalisé un logiciel ludique de sensibilisation
à la recherche opérationnelle, permettant au débutant de s'initier à des problèmes
complexe de manière amusante.

Ce projet fut pour nous l'occasion de travailler en groupe, et d'en découvrir au passage
les outils inhérents. Il nous a également permis de découvrir le langage
\emph{Python}, ainsi que la bibliothèque \emph{Pygame}.
Mais le point le plus important est sûrement que nous avons pu appliquer
bon nombre de préceptes appris lors de nos deux années passées en DUT.

Dans un souci de réalisme au niveau des délais, nous avons sélectionné de
manière arbitraire les problèmes traités, en choisisant ceux qui nous
semblaient les plus intéressants, avec des applications
concrètes instinctives et évidentes dans la vie réelle
(Comme par exemple le problème de plus court chemin dans les systèmes GPS).
On peut citer, entre autres, au rang des absents les problèmes de colorations de graphes.

Parmi les problèmes rencontrés, les plus importants ont eu trait
non pas aux côtés techniques, mais plutôt du côté humain,
notamment au niveau de la répartition de la charge de travail.

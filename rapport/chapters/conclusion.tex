
Nous avons ainsi réalisé un logiciel éducatif de sensibilisation à la recherhce opérationnelle dans le cadre de notre projet tutorer. 
Cette plateforme va permettre à des non-initié de découvrir  des problèmes complexes en s'amusant. 

Ce projet fut l'occasion de travailler en groupe, et d'utiliser des outils adaptés qui nous on permis de facilité notre travail en groupe. 
Il a également permis de découvrir le langage \emph{Python}, ainsi que la bibliothèque \emph{Pygame}. Il nous a aussi permis de mettre 
en application les connaissance qui nous ont été inculquées au cours de ces deux années de DUT. 

Toutefois nous aurions pu développer notre logiciel notamment en implémentant d'autre problème 
telle que le problème de colorisation d'un carte. Cependant, nous avons préferer faire un choix sur le nombre de problème à présenter.  
En effet, il n'était pas possible de présenter tout les problèmes d'optimisation. Nous avons privilégié des problèmes que l'on peut facilement modéliser dans la 
vie courante comme le problème du \emph{plus court chemin} qui peut être utiliser dans les GPS.

Au court de ce projet, nous avons rencontré quelques problèmes notamment des problèmes de répartiton du travail. 

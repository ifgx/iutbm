Dans le cadre de notre projet sous tutelle, nous avec réalisé un logiciel ludique de sensibilisation
à la recherche opérationnelle, permettant au profane de s'initier à des problèmes
complexe de manière amusante.

Ce projet fut pour nous de travailler en groupe, et d'en découvrir au passage
les outils inhérents. Il nous a également permis de découvrir le langage
\emph{Python}, ainsi que la bibliothèque \emph{Pgame}.
Mais le point le plus important est sûrement que nous avons pu appliquer
bon nombres de preceptes appris lors de nos deux années passées en DUT.

Dans un soucis de réalisme au niveau des délais, nous avons selecitonné de
manière arbitraire les problèmes traités, en choisisant ceux qui nous
semblaient le plus terre-à-terre, avec des applications
concrètes instinctives et evidente dans la vie réelle
(Comme par exemple le problème de plus court chemin dans les systèmes GPS).
On peut citer, entre autre, au rang des absents les problèmes de colorations de graphes.

Parmis les problèmes rencontrés, les plus importants ont eu trait
non pas aux côtés techniques, mais plutôt du côté humain,
notament au niveau de la répartition de la charge de travail.

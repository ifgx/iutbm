\section{Algorithmes}
    \subsection{Voyageur de commerce}
        \subsubsection{Génération des données}
            La génération des données se fait en deux temps :
            \begin{enumerate}
                \item Génération de points à des coordonnées aléatoires
                \item Génération de la matrice des distances
            \end{enumerate}
        \subsubsection{Interface}
            L'utilisateur sélectionne les points par simple clic,
            et peut annuler ses actions par clic droit.
            A chaque point sélectionné, le programme
            affiche la distance totale entre les points reliés.
            Dès lors qu'un point a été sélectionné, le programme
            affiche la distance minimum calculée, au-dessus de
            la distance actuelle parcourue par l'utilisateur.
            Lorsque tout les points ont étés reliés entre eux,
            le programme raccorde automatiquement le premier
            et le dernier point, afin de former une boucle.
        \subsubsection{Résolution}
            Lors de la sélection du premier point,
            le programme calcule à l'aide de l'algorithme
            du \emph{plus proche voisin} le chemin le plus court.
            Une fois que l'utilisateur a relié tout les points
            entre eux, le programme affiche la solution calculée
            en surimpression de celle proposée par l'utilisateur.
        \subsubsection{Algorithme}
        \begin{lstlisting}
computed_path = [self.matrix[0][ligne]]
self.matrix[0][ligne].visite = vrai
tmp = -1
max_distance = sqrt((max_x - min_x)^2 + (max_y - min_y)^2)
pour i de 1 a matrix[0][0] + 1
    minimum = max_distance
    pour j de 1 a matrix[0][0] + 1
        si 0 < matrix[ligne][j] < minimum
            si matrix[ligne][j].visite = faux
                minimum = matrix[ligne][j]
                tmp = j
            finsi
        finsi
    finpour
    si minimum != max_distance
        computed_path = computed_path + matrix[0][tmp]
        matrix[0][tmp].visite = vrai
        computed_len = computed_len + minimum
        ligne = tmp
    finsi
finpour
comuted_len = computed_len + matrix[1][tmp]
        \end{lstlisting}


	\subsection{Couplage}
		\paragraph{Contexte}
			Le probleme de couplage est modélisé par un jeu de choix.
			Il y a plusieurs clients, avec leurs préférences qui ont
			 le choix entre plusieurs pizzas, et le joueur doit
			 satisfaire un maximum de clients.
		\paragraph{Moteur}
			Au demarrage de l'application, le nombre de pizzas et
			 de clients est généré aléatoirement, dans un intervalle
			 assez bas, par soucis de complexité.
			Les préférences de chaque client sont aussi générés aléatoirement,
			 mais dans l'optique qu'un client ait au minimum une préférences.
		\paragraph{Interface}
		    L'écran est divisé en trois colonnes :
		    \begin{itemize}
		        \item[à gauche] les préférences de chancun des clients
		        \item[au centre] les clients
		        \item[à droite] les pizzas disponibles
		    \end{itemize}

		L'utilisateur \emph{sélectionne} un client en cliquant dessus.
		    Une fois un client sélectionné, il est possible
		    d'associer une pizza à ce client, en cliquant dessus.
		    Si la pizza est déjà attribuée, ou si le client a déjà
		    une pizza qui lui est affectée, un message d'erreur s'affiche.
		    Si ni le client n'a de pizza, ni la pizza de client, l'association
		    se fait, et est représentée par un trait.
		L'utilisateur a la possibilité d'annuler ses choix en cliquant avec le bouton
		    droit de la souris (ce qui annule la dernière action).
		\paragraph{Résolution}
			L'algorithme permet d'avoir la solution comportant un maximum de clients
			 satisfait et de verifier si la solution proposée par le joueur correspond
			 a la meilleure solution possible.
 			Si l'utilisateur n'a pas trouvé la solution optimale, la solution s'affiche
			 en surimpression sur le choix du joueur.

		\paragraph{Algorithme de résolution}
			Tout d'abord, chaque sommet représentant les personnes est lié a un sommet
			 Source, par des arrêtes de capacité 1, ce qui permet une unique satisfaction
			 des personnes.
			De même pour les objets, ce qui permet l'unique utilisation d'un objet.
			Ensuite, une liste des éléments nécessaire a l'algorithme :
		        \begin{enumerate}
		        	\item lp : liste des sommets concernant les personnes
		        	\item lo : liste des sommets concernant les objets
				\item la : liste des couples de sommets correspondant aux arrêtes, et donc aux préférences
				\item nboccurence(i) : fonction qui retourne les sommet dont le degré sortant est de i
				\item associera(s2) : retourne les sommets associé a s2
				\item capacite(s) : retourne vrai si l'on peut atteindre s a partir du sommet Source
				\item maxoccur() : retourne le degré sortant maximum du graphe
				\item supprimercouple : supprimer un couple d'une liste de couples
            		\end{enumerate}

			\emph{L'algorithme :}

			\begin{lstlisting}
			solution = []
			tantque taille(la) != 0 faire
				pour i allant de 1 a maxoccur() + 1
					pour pers dans nboccurence(i)
						pour obj dans associera(pers)
							si capacite(pers) alors
								solution = [pers,obj]
								la.supprimercouple([pers,obj])
							sinon
								la.supprimercouple([pers,none])
							finsi
						finpour
					finpour
				finpour
			fintantque
			\end{lstlisting}





    \subsection{Sac à Dos}
        \paragraph{Contexte}
        On pourra modéliser ce problème par un tableau contenant les objets et leur
        valeur en vis-à-vis. L'utilisateur fera glisser les objets du tableau vers
        le sac à dos. Un indicateur permettra de visualiser le poids actuel total
        des objets dans le sac, le poids maximal que peut supporter le sac, et le
        poids restant (différence entre ces deux résultats).

            Dans ce mini-jeu, le but du joueur est d'aider un pizzaiolo à
            sélectionner des ingrédients à transporter à un concours de pizza.
            Il ne peut pas prendre tout ce qu'il souhaite et doit faire des
            choix entre qualité et quantité : de bons ingrédients lui donnera
            plus de points par le jury, cependant la pizza présenté a un poids
            limité au gramme près.
        \paragraph{Moteur}
            Au démarrage de l'application, une liste d'ingrédients est créée.
            Pour respecter une certaine logique, on ne peux pas générer des
            chiffre aléatoirement. (Il semblerait assez illogique qu'une
            garniture au fromage pèse plusieurs centaines de grammes par
            exemple.)
        \paragraph{Interface}
            L'interface se présente en 2 colonnes. La première, à gauche,
            contient les objets disponible pour la pizza. La seconde, à droite,
            montre le contenu actuel de la pizza que le joueur à choisi.
            Un clic sur une icone d'un ingrédients permet de l'échanger entre
            les deux colonnes.
            Lorsque l'utilisateur a terminé, il clique sur le bouton de
            validation pour savoir si il a trouvé la solution correcte.
        \paragraph{Résolution}
            La résolution du problème s'effectue grâce à un calcule de la solution optimale
            par programmation dynamique. Notons $V(k,y)$ la valeur optimale au problème du sac à dos
            réduit au $k$ premier objet avec un sac de poids maximale de $y$.
            $y_{k}$ le poids de l'objet k et $v_{k}$ la valeur de l'objet $k$.

            Si $ y - y_{k} >= 0 $

                $V(k,y) = max\{ V(k-1,y)  ,v_{k} + V(k-1,y-y_{k}) \}$

            Sinon

                $V(k,y) = V(k-1,y)$

            On notera aussi que si il n'y a pas d'objet on a : $V(0,y) = 0$
            De plus si le poids est nul on a: $V(k,0) = 0$

	\emph{Algorithme}

\begin{lstlisting}
solution = []
pour i allant de 0 a nombre_Objet+ 1
    pour j allant de 0 a poids_maximum + 1
        si i != 0 et j != 0
            w = Objet[i].poids
            si j > w alors
                si solution[i-1][j].poids > Objet[i].valeur+
                            ...solution[i-1][j-w].valeur
                    solution[i][j] = solution[i-1][j]
                sinon
                    solution[i][j] = solution[i-1][j-w] + Objet[i]
                finsi
            sinon
                solution[i][j] = solution[i-1]
            finsi
        sinon
            solution[i][j] = 0
        finsi
    finpour
finpour
\end{lstlisting}


\subsection{Problème du plus court chemin}

        \paragraph{Contexte}
	    Dans ce mini-jeu vous travaillerez en tant que livreur de pizza. On vous a demander de 			livrer  une pizza à l'autre bout de la ville. Pour cela il vous faudra trouver le plus court chemin
	    pour vous rendre de la pizzeria jusqu'au lieu de la livraison.
        \paragraph{Moteur}
            Au démarrage du jeu, le programme génère un graphe
            représentant des lieux de la ville. Ces lieux sont reliées par des arcs pondérés.
            Le poids des arcs corresponds à la distance entre chaque lieux.

        \paragraph{Interface}

        L'utilisateur pourra cliquer sur les différents lieux, tour à tour,
        pour ce déplacer. Un compteur kilométrique indiquera la
        distance parcourue en temps réel.

        Le joueur peut effectuer un clic droit pour annuler son dernier
        mouvement.

        Pour revenir au début il suffit de cliquer sur la ville de départ.

        Le jeu continue tant que le joueur n'a pas trouvé le chemin
        le plus court. Il peut aussi cliquer sur \og solution\fg, qui
        affichera le chemin segment par segment.



        \paragraph{Résolution}
            Lorsque la ville de début et la ville finale sont reliées,
            on appelle la méthode de résolution par l'algorithme de \emph{Dijkstra}.

            Le programme affiche ensuite si le joueur à trouvé la solution optimale.

		\paragraph{Algorithme de Dijkstra} ~

			L'algorithme de Dijkstra est l'un des algorithme de résolution du problème du
			\emph{plus court chemin} les plus connus. Cette algorithme s'applique à des graphes pondérés orientés ou non.

			L'algorithme de \emph{Dijkstra} :
			\begin{lstlisting}
liste_sommets = [+oo pour i dans sommets]
S(liste_sommets[0]) = 0
tantque taille(liste_sommets) != 0
    element_courant = min(liste_sommets)
    mettre_a_jour(element_courant)
    supprimer element_courant de liste_sommets
fintantque
			\end{lstlisting}
			Pour mettre à jour un sommet i:

				Pour chaque autre sommet j qui est reliée à ce sommet i on vérifie si
				la valeurs de j est supérieur à la valeurs de i plus la valeurs l'arc entre i et j

			\begin{lstlisting}
mettre_a_jour(sommet):
pour chaque sommet_choisit dans graphe
    si sommet_choisit.visiter == faux
        si sommet_choisit.valeur > sommet.valeur + ...
            ... distance entre les deux sommets
		    sommet_choisit.valeur = sommet.valeur + ...
            ... distance entre les deux sommets
        finsi
	finsi
finpour
			\end{lstlisting}


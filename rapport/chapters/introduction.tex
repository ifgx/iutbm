% a bit hackish
\setcounter{footnote}{0}

Le but de ce projet tuteuré est de créer un logiciel éducatif de sensibilisation à la recherche
 opérationnelle. Ce logiciel permettra à des non-initiés (étudiants, entre autres) de s'initier
 de manière ludique aux problématiques de la recherche opérationnelle, qui font partie de notre
 vie de tous les jours.

Les applications pratiques de la recherche opérationnelles sont courantes. On pourra citer,
 entre autres:

\begin{itemize}
 \item Comment se rendre d'un point A à un point B en prenant le plus court chemin possible?
  (problème du plus court chemin);
 \item Comment répartir équitablement des objets à plusieurs personnes en en satisfaisant le
  plus grand nombre? (problème du couplage);
 \item Comment emporter avec soi un maximum d'objets de valeurs et poids différents, sachant
  que la capacité de notre sac est limitée? (problème du sac à dos);
 \item Comment passer par toutes les villes d'une zone donnée en optimisant au maximum le
  parcours? (problème du voyageur de commerce).
\end{itemize}


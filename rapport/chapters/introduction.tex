% a bit hackish
\setcounter{footnote}{0}

\section{But du projet}
Le but de ce projet est de créer un logiciel éducatif de sensibilisation à la recherche
opérationnelle. La recherche opérationnelle (aussi appelée aide à la décision) peut être définie
comme l'ensemble des méthodes et techniques rationnelles orientées vers la recherche de
la meilleure façon d'opérer des choix en vue d'aboutir au résultat visé ou au meilleur résultat possible.%
~\cite{ref.yrr}

Ce logiciel permettra à des non-initiés (étudiants entre autres) de s'initier
de manière ludique aux problématiques de la recherche opérationnelle, qui font partie de notre
vie de tous les jours.

\section{Notion de recherche opérationnelle}
La recherche opérationnelle (également appellée \emph{aide à la decision})
est la recherche de la meilleure façon d'aboutir au meilleur résultat
possible, ou à un résultat visé. Le domaine fait appel au raisonnement 
mathématique (logique, probabilités, analyse de données) et à la modélisation 
des processus. 

\subsection{Historique}
La recherche opérationnel à été étudier dès le XVII\up{ème} par des mathématiciens tel que Blaise Pascal qui 
a tenter de résoudre des problèmes de décision dans l'incertain avec l'espérance mathématique. 
Au début du XX\up{ème} siècle, l'étude de la gestion de stock peut être considérée comme étant 
à l'origine de la recherche opérationnelle moderne notamment avec la formule du lot économique 
proposée par Harris en 1913. 

Mais c'est lors de la Seconde Guerre mondiale que la pratique va s'organiser . 
En 1940, Patrick Blackett dirigera la première équipe de recherche opérationnelle, 
pour résoudre certains problèmes tels que l'implantation optimale de radars de surveillance 
ou la gestion des convois d'approvisionnement. Après la guerre, les techniques sont considérablement améliorés
notamment grâce à l'explosion des capacités de calcul des ordinateurs.
\section{Notion de complexité}
\subsection{Émergence de la notion}
La \emph{Théorie algorithmique de l'information} fut initiée par Kolmogorov~\cite{ref.kol}, Solomonov%
~\cite{ref.sol} et Chaitin~\cite{ref.cha}
dans les années 1960, et vise à quantifier et qualifier le contenu en information d'un ensemble de données.
C'est cette théorie qui a permis de formaliser la notion de complexité qui nous intéresse ici,
en considérant qu'un objet est d'autant plus complexe qu'il nécessite d'informations pour être décrit.
\subsection{La théorie de la complexité}
La théorie de la complexité des algorithmes étudie de manière formelle la quantité de ressources
temporelles et spatiales (la quantité de mémoire requise par exemple) nécessaire
par l'exécution d'un programme.
\subsection{Notation} La complexité d'un algorithme se note en général en utilisant
la notation $O$.
Par exemple, le parcours d'une liste de 700 éléments demandera au pire 700 étapes.
La complexité dans le pire des cas pour n entrées se note $O(n)$, ce qui signifie que dans le pire
des cas, le temps de calcul est de l'ordre de grandeur de n : il faut parcourir tous les n
éléments une fois.

\begin{center}
    \begin{table}[h]
        \begin{tabular}{| l | c | c |}
            \hline
            Notation & Type & Exemple \\
            \hline
            $O(1)$ & complexité constante & accès dans un tableau\\
            $O(n)$ & complexité linéaire & parcours de liste\\
            $O(n log(n))$ & complexité linéarithmique & tri optimal d'un tableau\\
            $O(n^{2})$  & complexité quadratique & parcours de tableaux 2D\\
            $O(n!)$ & complexité factorielle & problème du voyageur de commerce\\
            \hline
        \end{tabular}
        \caption{Exemple de complexités~\cite{ref.table}}
    \end{table}
\end{center}



Le projet s'articulera autour d'une application interactive.
L'application proposera un menu clair, permettant
d'acceder rapidement aux problèmes proposés.

Les jeux se dérouleront par étapes.
\begin{enumerate}
    \item Un texte explicatif du problème posé
    \item Le joueur essaye de trouver la solution optimale par lui-même
    \item Le programme compare la solution calculée à celle du joueur
\end{enumerate}

\section{Portabilité et accessibilité}
    Le programme devra être portable sur les principales plateformes
    grand public (\emph{Windows}, \emph{GNU/Linux}, et \emph{Mac OSX}).
    Il devra également être accessible, le public cible n'étant
    pas forcément familier de l'outil informatique.

\section{Langue anglaise}
    Afin de permettre une diffusion la plus large possible,
    l'application devra être écrite en anglais.

\section{Thème unifié}
    Afin de rendre l'application ludique, elle devra
    être habillée d'un thème unifié,
    afin de montrer les \emph{problèmes} sous un
    jour \emph{amusant}.

\section{Menu principal}
    Le menu principal doit permettre d'accéder rapidement
    aux algorithmes développés, en étant le plus simple possible.

\section{Notion de complexité}
La théorie de la complexité s'attache à connaître la difficulté (ou la
complexité) d'une réponse par algorithme à un problème, dit algorithmique, posé
de façon mathématique. Pour la définir, il faut présenter les concepts de
problèmes algorithmiques, de réponses algorithmiques aux problèmes, et la
complexité des problèmes algorithmiques.~\cite{wp-complex} 
Cette complexité va permettre d'évaluer différents algorithmes qui vont 
résoudre le même problème. 
 Il existe de nombreux algorithmes pour trier un tableau. On va présenter deux 
de ces algorithmes l'un étant le tri par insertion, l'autre étant le tri rapide dit \emph{quik sort}.
Le tri par insertion a une complexité polynomiale qui est égale à $O(n)$ . 
Le tri rapide lui a une complexité linéarithmique qui est égale à $O(n \times \log(n) )$ . 
Le tri rapide est beaucoup plus efficace que le tri à bulle sur de  grand nombre. 
Toutefois sur des tableux des moins de 15 élément le tri par insertion est plus efficace. 
Ici la complexité permet de choisir un algorithme selon les type de données que l'on va traiter.

\section{Notion de recherche opérationnelle}
La recherche opérationnelle (également appellée \emph{aide à la decision})
est la recherche de la meilleure façon d'aboutir au milleur résultat
possible, ou au résultat visé.

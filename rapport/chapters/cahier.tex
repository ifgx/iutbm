Le projet s'articulera autour d'une application interactive.
  L'utilisateur aura à sa disposition trois sous-menus, chacun comportant
  plusieurs jeux basés sur des algorithmes de plus en plus difficiles.
  Ces jeux seront classés selon leur niveau de difficulté -- d'un point de
  vue algorithmique~:

Les jeux se dérouleront par étapes.
\begin{enumerate}
    \item Un texte explicatif du problème posé
    \item Le joueur essaye de trouver la solution optimale par lui-même
    \item le programme compare la solution calculée à celle du joueur
\end{enumerate}

\section{Portabilité et accessibilités}
    Le programme devra être portable sur les principales plateformes
    grand public (\emph{Windows}, \emph{GNU/Linux}, et \emph{Mac OSX}).
    Il devra également être accessible, le public cible étant
    l'utilisateur lambda.

\section{Menu principal}
    Le menu principal doit permettre d'acceder rapidement
    aux algorithmes developpés, en étant le plus simple possible.

\section{Algorithmes en temps polynomiaux}
    \subsection{Le plus court chemin}
        Le but du jeu du plus court chemin est d'aider un personnage à se
        rendre d'un point a à un point b en empruntant le chemin le plus
        court possible.

        l'utilisateur pourra cliquer sur les différentes villes, tour à tour,
        pour déplacer le personnage. un compteur kilométrique indiquera la
        distance parcourue.

        le joueur peut effectuer un clic droit pour annuler son dernier
        mouvement.

        le jeu continue tant que le joueur n'a pas trouvé le chemin
        le plus court. il peut aussi cliquer sur \og solution\fg, qui
        affichera le chemin segment par segment.

        Les niveaux sont générés aléatoirement.

    \subsection{Le couplage}
        Le but du jeu est de partager un certain nombre d'objets avec plusieurs
        personnes. Chaque personne doit dresser une liste de ses objets préférés,
        et l'on doit distribuer de manière équitable les objets en essayant de
        satisfaire le plus grand nombre.

        L'utilisateur pourra choisir différents niveaux; le nombre d'objets
        changera en fonction du niveau de difficulté.

\section{Algorithmes pseudo-polynomiaux}
    \subsection{Le problème du sac à dos}
        Ce problème consiste à emporter des objets de valeurs différentes. Il s'agit
        d'optimiser la valeur des objets que l'on peut prendre en veillant à ne pas
        dépasser la capacité du sac à dos.

        On pourra modéliser ce problème par un tableau contenant les objets et leur
        valeur en vis-à-vis. L'utilisateur fera glisser les objets du tableau vers
        le sac à dos. Un indicateur permettra de visualiser le poids actuel total
        des objets dans le sac, le poids maximal que peut supporter le sac, et le
        poids restant (différence entre ces deux résultats).

        On pourra scénariser ce jeu avec, par exemple, une histoire où un personnage
        doit quitter son château assiégé en emportant un maximum d'objets de valeur.

\section{Algorithmes en temps exponentiels}
     \subsection{Le problème du voyageur de commerce}
            Un voyageur de commerce se situe dans une certaine ville A. Il doit passer
            par toutes les villes de la carte pour revenir à son point de départ. Le but
            est de déterminer le plus court chemin entre toutes les villes. On pourra
            reprendre le concept du premier jeu, en modifiant les distances entre les
            villes.

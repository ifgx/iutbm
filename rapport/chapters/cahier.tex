Le projet s'articulera autour d'une application interactive.
  L'utilisateur aura à sa disposition trois sous-menus, chacun comportant
  plusieurs jeux basés sur des algorithmes de plus en plus difficiles.
  Ces jeux seront classés selon leur niveau de difficulté -- d'un point de
  vue algorithmique~:

Les jeux se dérouleront par étapes.
\begin{enumerate}
    \item Un texte explicatif du problème posé
    \item Le joueur essaye de trouver la solution optimale par lui-même
    \item Le programme compare la solution calculée à celle du joueur
\end{enumerate}

\section{Portabilité et accessibilité}
    Le programme devra être portable sur les principales plateformes
    grand public (\emph{Windows}, \emph{GNU/Linux}, et \emph{Mac OSX}).
    Il devra également être accessible, le public cible étant
    l'utilisateur lambda.

\section{Langue anglaise}
    Afin de permettre une diffusion la plus large possible,
    l'application devra être écrite en anglais.

\section{Thème unifié}
    Afin de rendre l'application ludique, elle devra
    être habillée d'un thème unifié,
    afin de montrer les \emph{problèmes} sous un
    jour \emph{amusant}.

\section{Notion de complexité}
TODO

\section{Menu principal}
    Le menu principal doit permettre d'accéder rapidement
    aux algorithmes développés, en étant le plus simple possible.


\section{Langage}
    Le programme est entièrement codé en Python,
    et utilise la bibliothèque \emph{pygame}.
    \subsection{Python}
        Le \emph{Python} est un langage de programmation de haut niveau\footnote{Qui fait
        abstraction des détails de la machine sur laquelle il s'exécute} interprété
        \footnote{Au contraire des langages compilés, le python est évalué
        lignes par lignes par un \emph{interpréteur}, au lieu d'être exécuté directement
        par la machine}. Il fonctionne sur la plupart des plateformes informatiques.
    \subsection{pygame}
        pygame est une bibliothèque écrite en \emph{Python} initialement prévue pour
        le développement de jeux vidéo. Elle permet de programmer la partie multimédia
        (graphismes, sons, entrées des périphériques (clavier, couris, joystick,...),
        système de fenêtrage, ...) rapidement et facilement).

\section{Outils}
    Tout les outils utilisés, sans exception sont disponibles sous
    licence libre.
    \subsection{git}
        \emph{git} fût utilisé en gestionnaire de version,
        afin de simplifier le travail collaboratif.
        Il est également un excellent indicateur de la répartition
        de la charge de travail.
    \subsection{pep8, pylint}
        \emph{pep8}, et \emph{pylint} sont tout les deux
        des outils d'analyse statique de code \emph{Python}
        \subsubsection{pep8}
            \emph{pep8} se charge de vérifier si le code qui lui
            est fourni en entrée respecte bien les convention de codage
            python de la norme \emph{pep8} (le logiciel est homonyme).
        \subsubsection{pytlint} pylint est quant à lui un outil d'analyse
            statique pur et dur. Il vérifie grossièrement la sémantique du code,
            ainsi que sa présentation, et son respect des bonnes pratiques.
            Par exemple, si les classes\footnote{Objet logiciel representant une entité}
            possèdent bien des commentaires les décrivant, ou encore si les modules
            \footnote{Un module est une \emph{brique} logicielle permettant
            de s'appuyer sur l'existant, et donc de ne pas réinventer la roue}
            importés sont bien utilisés, si les variables ont des noms corrects, ...
        \subsubsection{vim}
            Ce projet fût également l'occasion pour certain de découvrir \emph{vim},
            un puissant éditeur de texte modal.

\section{Choix du thème}
Le programme adopte un thème unifié autour du monde de la pizza. Dans tous les mini-jeux, on pourra
ainsi retrouver soit une livraison de pizza, soit une distribution de pizza. Cette unification permet à
l'utilisateur de mieux se retrouver au sein des différents mini-jeux. De plus, elle procure aux mini-jeux
une certaine cohérence qui aide à la compréhension.

\section{Menu principal}
    Le menu principal est un menu circulaire,
    permettant de choisir les algorithmes, et de
    quitter le programme.

\documentclass[12pt]{report}

\usepackage[utf8]{inputenc}
\usepackage[francais]{babel}
\usepackage[T1]{fontenc}
\usepackage{lmodern}

\setcounter{secnumdepth}{3}

\renewcommand\thesection{\Roman{section}.}
\renewcommand\thesubsection{\arabic{subsection})}
\renewcommand\thesubsubsection{\arabic{subsection}.\arabic{subsubsection})}

\begin{document}
  \vspace*{\fill}
  \begin{center}
    {\bf \Large Cahier des charges}
  
    % pas très sexy, je suis preneur de mieux
    \vspace{3em}
  
    {\large Projet tutoré - DUT Informatique}
  
    \vspace{5em}
  
    BENOIT Marine \\
    DUROSIER Florian \\
    NOWINSKI David \\
	SID Ali \\
    VOISIN Julien \\
    ZAMBELLI Cédric
  \end{center}
  \vspace*{\fill}
  \newpage
  
  \tableofcontents
  
  \newpage
  
  \section{Pourquoi ce projet~?}
  
  Le but est de créer un logiciel éducatif de sensibilisation à
   la recherche opérationnelle. Ce logiciel permettra à des non-initiés
   (étudiants, entre autres) de s'initier de manière ludique aux
   problématiques de la recherche opérationnelle, qui font partie
   de notre vie de tous les jours.
 
 \section{Comment~?}
 
 Le projet s'articulera autour d'une application interactive.
  L'utilisateur aura à sa disposition trois sous-menus, chacun comportant
  plusieurs jeux basés sur des algorithmes de plus en plus difficiles.
  Ces jeux seront classés selon leur niveau de difficulté -- d'un point de
  vue algorithmique~:
 
 \begin{itemize}
   \item les algorithmes polynomiaux,
   \item les algorithmes pseudo-polynomiaux,
   \item les algorithmes exponentiels.
   \end{itemize}
 
 Le joueur essayera de trouver par lui-même les solutions aux différents
  problèmes proposés. La solution lui sera présentée de manière animée
  (par exemple, pour le plus court chemin, une recherche animée).
 
 Les jeux continuent tant que le joueur n'a pas trouvé la solution idéale.
  Un bandeau, présent en haut de la fenêtre de jeu, initialement rouge avec
  un texte blanc \og solution optimale non trouvée \fg, passera au vert
  lors de la résolution (le texte changera en \og vous avez trouvé, bravo~!\fg).

 Le projet sera programmé avec le langage Python, afin d'assurer une portabilité
  vers tous les systèmes d'exploitation.

 \section{Description fonctionnelle}
 
 \subsection{La page d'accueil}
 
  Elle doit comporter des liens ou boutons vers les trois catégories
   précédemment vues. Le design doit être attrayant et inviter les
   utilisateurs (qui sont, on le rappelle, supposés être novices et
   n'avoir que pas ou peu de connaissances en algorithmique et en
   recherche opérationnelle / optimisation) à s'intéresser au sujet.
  
  \subsection{La partie interactive}
  
  \subsubsection{Temps polynomiaux}
  
  \paragraph{Le plus court chemin}
  
  Le but du jeu du plus court chemin est d'aider un personnage à se
   rendre d'un point A à un point B en empruntant le chemin le plus
   court possible.
  
  L'utilisateur pourra cliquer sur les différentes villes, tour à tour,
   pour déplacer le personnage. Un compteur kilométrique indiquera la
   distance parcourue.
  
  Le joueur peut effectuer un clic droit pour annuler son dernier
   mouvement.
  
  Le jeu continue tant que le joueur n'a pas trouvé le chemin
   le plus court. Il peut aussi cliquer sur \og solution\fg, qui
   affichera le chemin segment par segment.
  
  Les niveaux sont générés aléatoirement.
  
  \paragraph{Le couplage}
  
  Le but du jeu est de partager un certain nombre d'objets avec plusieurs
   personnes. Chaque personne doit dresser une liste de ses objets préférés,
   et l'on doit distribuer de manière équitable les objets en essayant de
   satisfaire le plus grand nombre.

  L'utilisateur pourra choisir différents niveaux~; le nombre d'objets
   changera en fonction du niveau de difficulté.

  \subsubsection{Temps pseudo-polynomiaux}
  
  \paragraph{Le problème du sac à dos}
  
  Ce problème consiste à emporter des objets de valeurs différentes. Il s'agit
   d'optimiser la valeur des objets que l'on peut prendre en veillant à ne pas
   dépasser la capacité du sac à dos.

  On pourra modéliser ce problème par un tableau contenant les objets et leur
   valeur en vis-à-vis. L'utilisateur fera glisser les objets du tableau vers
   le sac à dos. Un indicateur permettra de visualiser le poids actuel total
   des objets dans le sac, le poids maximal que peut supporter le sac, et le
   poids restant (différence entre ces deux résultats).

  On pourra scénariser ce jeu avec, par exemple, une histoire où un personnage
   doit quitter son château assiégé en emportant un maximum d'objets de valeur.

  \paragraph{Meilleur rendement dans une confiserie}
  
  Dans ce jeu, l'utilisateur doit réussir à satisfaire au mieux les désirs
   d'une confiserie. La confiserie fabrique trois types de bonbons: des
   caramels, des sucre d'orge et des chewing-gum. Chaque bonbon se vend à
   un prix qui lui est propre. On doit fabriquer une quantité minimale
   de chaque bonbon, mais on ne peut pas dépasser une certaine quantité
   maximale. On dispose également d'un certain temps imparti pour réaliser
   tous les bonbons; chaque bonbon ayant évidemment un temps de fabrication
   différent.

  La modélisation pourra s'effectuer à l'aide d'un tableau à double entrée
   présentant, en ligne, un bonbon et, en colonne, les différentes
   caractéristiques (temps de fabrication, nombres minimal et maximal).
   L'utilisateur devra cliquer sur les bonbons pour les fabriquer, une jauge
   de temps en haut de l'écran diminuera au fur et à mesure que le temps
   de fabrication des bonbons s'écoule. Là encore, l'utilisateur pourra
   revenir en arrière en effectuant un clic sur le bouton droit.

  \subsubsection{Temps exponentiels}
  
  \paragraph{Le problème du voyageur de commerce}
  
  Un voyageur de commerce se situe dans une certaine ville A. Il doit passer
   par toutes les villes de la carte pour revenir à son point de départ. Le but
   est de déterminer le plus court chemin entre toutes les villes. On pourra
   reprendre le concept du premier jeu, en modifiant les distances entre les
   villes.

  \section{Équipe}
  
  \begin{itemize}
    \item Marine Benoit
    \item Florian Durosier
    \item David Nowinski
    \item Baptiste Valthier
    \item Julien Voisin
    \item Cédric Zambelli
  \end{itemize}
  
  Cette équipe risque d'être dissoute d'une part par d'éventuels redoublements
   de semestre 2, et d'autre part, par le départ de Baptiste Valthier à
   l'Université de Portsmouth dans le cadre d'un échange Erasmus.
  
  \section{Méthode de travail et organisation}
  
  Dans un premier temps, s'effectuera une phase de recherche d'informations
   et de documentation:
  
  \begin{itemize}
  	\item documentation sur la recherche opérationnelle;
  	\item documentation sur les différents algorithmes utilisés;
  	\item recherches sur les librairies graphiques en Python;
  	\item approche de l'utilisation d'un logiciel de gestion de versions
  	 (tel que {\tt mercurial}).
  \end{itemize}
  
  Le squelette de l'application fera l'objet d'une implémentation commune.
   Nous discuterons ensemble de la librairie graphique utilisée, des différentes
   conventions de codage et réaliserons ensemble le menu.
  
  Les algorithmes seront répartis en groupes de deux personnes. Une fois
   le codage terminé, pourront s'ensuivre des échanges de code et des tests.
  
  \section{Objectifs généraux}
  
  L'objectif est de créer une application \emph{didactique}, qui s'articulera
   autour~:
  
  \begin{itemize}
  	\item d'un {\bf menu général};
  	\item de solutions {\bf pas-à-pas};
  	\item d'une génération {\bf aléatoire} des jeux;
  	\item de différents {\bf niveaux} de difficulté.
  \end{itemize}
  
  L'application devra être portable et multiplate-forme. Le langage de
   programmation retenu est le Python.

  \section{Solutions alternatives}
  
  Si nous échouons à notre tâche, nous pourrons au choix supprimer certains
   algorithmes jugés trop difficiles à mettre en place, ou négliger l'aspect
   graphique afin de consacrer nos efforts au programme en lui-même.

  \section{Contrainte}
  
  Nous nous fixons la contrainte de terminer le projet en la période
   d'un semestre.
  
\end{document}
​
